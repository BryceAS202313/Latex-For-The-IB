Adding parameters to a macro is very easy. After the macro to define, just enter the number of parameters in brackets. To use the parameters, just use a number sign and then the number of the parameter. Here is an example of a macro using two parameters \verb=\newcommand{\hello}[2]{Hello #1, my name is #2.}=. This macro can then be used by typing \verb=\hello{Sarah}{Bryce}= with the result of \newcommand{\hello}[2]{Hello #1, my name is #2.}``\hello{Sarah}{Bryce}'' (quotation marks not included). There are more advanced ways of specifying parameters which is included in the \hyperref[section:programming/advancedFeatures/plainTeX]{Plain \TeX{}} section under \hyperref[section:programming/advancedFeatures]{Advanced features}.
