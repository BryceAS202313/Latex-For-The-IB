In general, \LaTeX{} will expand macros in the order that it sees them. This means that if you have a macro as an argument to another macro, that second macro will see a macro as its argument rather than whatever that macro expands to. This is useful is some cases. For example, I have shown macros in this document so far using the \verb=\verb= command. If \LaTeX{} were to expand in a different order, the commands I wish to show you would be expanded before the verb command changed them to plain typewriter text. \par
There are some cases where this is not wanted. Perhaps you are trying to define a macro with a name based on the output of another macro. The explanation for how this works will be in the next section but it is important to note that this involves another macro which must be expanded before the macro to define the macro is expanded. It's easy to tell \LaTeX{} to expand the macro after the next one first using the \verb=\expandafter= command. It is also possible to chain these together, if \verb=\expandafter= is the target of the first \verb=\expandafter= command, then it will expend an even further macro first. This can be chained for quite a while before memory runs out. \par
Another important place where this expansion order comes into play is with macro definition itself. When a macro is defined, nothing in its definition is expanded. This means that if you define a macro as essentially an alias for another macro and then redefine the other macro, your original definition will change too. There are various ways to get around this which is covered later. \par
There is one issue that you may come across. Not all macros are expandable. The main reason for this is if it is a raw \TeX command which does not expand into anything but simply preform an action. Few of these have been discussed so far but they will come up later. There are a few other packages which have specific commands that have these issues as well. For example, the \texttt{xString} package states this in the documentation \texttt{The macros of this package are not purely expandable}. That package does provide ways to get around that issue however.
