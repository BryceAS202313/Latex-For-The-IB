\section{What is \LaTeX?}
\LaTeX{} is a typesetter. Its main job is to take plain text as an input and return a PDF. You are probably used to using a program such as Word or Pages to produce documents. When you use these, you directly manipulate the formatting of the text as well as the text itself. What makes \LaTeX{} different is that everything is written as plain text. All of the formatting is added using macros. Some macros take arguments allowing for conditional logic. There are many packages available that add complex macros allowing for everything from the most basic of tables to plots with trend lines to complex dynamically created graph theory.\par
Another important part of \LaTeX{} is equation typesetting. Equations are typeset in a way as natural as if it was written by hand. Nested fractions, dynamically sized parenthesise, and just about any advanced function imaginable is available either by default or though some package.

\section{Installing \LaTeX}
There are a number of online \LaTeX{} editors. The most popular of them is \href{https://www.overleaf.com}{Overleaf}. Overleaf also has some useful \LaTeX{} \href{https://www.overleaf.com/learn}{documentation}.\par
% Add alternatives as I find out
There are also a number of OS-specific implementations detailed below.

\subsection{macOS and iOS}
The most popular standalone application for macOS and iOS is TexPad which costs approximately \SI{30}[\$]{} for the desktop version and \SI{15}[\$]{} for the mobile version. There is also Latex Presentation for presenting sideshows made with the Beemer document style. MathKey allows you to draw an equation and have the \LaTeX{} code appear automatically (also available on iOS). There are a number of other less popular applications available as well.\par
It is also possible to download and the \LaTeX{} packages manually and compile .tex documents yourself. This allows you to edit those documents in any text editor. This further enables the production complex build script and other advanced features that will be covered in a later chapter. The macOS version is called \href{http://www.tug.org/mactex/}{MacTeX}. It can either be installed directly though their website or via HomeBrew.\par
There are a few other iOS applications available. If you want to test small things out, not full documents (though they are supported partially), VerbTeX works well. It does not allow including or imputing files and unless you save the PDF elsewhere, you must compile the PDF every time you want to switch between it and the code. There is a ``Pro'' version that costs money and removes these limitations though it is still not the best. The final good option is TeXWriter which is comparable to TexPad.

\subsection{Windows}
There are a number of \LaTeX{} editors available for Windows, the main ones are \href{https://miktex.org/}{MiKTeX}, \href{http://www.tug.org/protext/}{proTeXt}, and \href{http://www.tug.org/texlive/}{TeX Live}. TeX Live installs just the parser itself which simply takes in .tex documents and spits out pdfs. This allows you to edit those documents in any text editor. This further enables the production complex build script and other advanced features that will be covered in a later chapter.

\subsection{Linux}
Most Linux distributions have some form of \LaTeX{} package available in their package manager. This has different names in different names for each distribution. Note that these are command line tools that take .tex files and produce pdfs. This allows you to edit those documents in any text editor. This further enables the production complex build script and other advanced features that will be covered in a later chapter. Most Linux users are probably familiar with command line tools however if a GUI is desired it will be necessary to use an online editor.