\section{Programming and Advanced LaTeX}
% What does it mean to program in LaTeX

\subsection{Basic Macros}
There are a number of ways to define new macros. The most basic is the \verb=\newcommand= macro. This takes two arguments, the name of the new macro and what it expands to. Say I wanted to define a basic macro that expands to some text. The basic way to do this would be something like as follows. \verb=\newcommand{\hello}{Hello, World!}=. Because a macro is considered to be only a single value, the first pair of braces are actually not required though it can make the definition more confusing to read. If you were to remove those braces then the definition would look like \verb=\newcommand\hello{Hello, World!}=. This macro only allow defining a new macro, if you wish to redefine a preexisting macro, whether defined ourselves or by another package, a different command must be used which is \verb=\renewcommand=.
\subsubsection{Parameters}
Adding parameters to a macro is very easy. After the macro to define, just enter the number of parameters in brackets. To use the parameters, just use a number sign and then the number of the parameter. Here is an example of a macro using two parameters \verb=\newcommand{\hello}[2]{Hello #1, my name is #2.}=. This macro can then be used by typing \verb=\hello{Sarah}{Bryce}= with the result of \newcommand{\hello}[2]{Hello #1, my name is #2.}``\hello{Sarah}{Bryce}'' (quotation marks not included). There are more advanced ways of specifying parameters which is included in the \hyperref[section:programming/advancedFeatures/plainTeX]{Plain \TeX{}} section under \hyperref[section:programming/advancedFeatures]{Advanced features}.
\subsubsection{Optional Parameters}
It is possible to add a single optional parameter using the command we have used before. This is used be declaring the total number of parameters and then adding a default value for the first one. An example of this would be:
\begin{verbatim}
    \newcommand{\hello}[2][world]{Hello #1, my name is #2.}
\end{verbatim}
If the command is used with just the one parameter, then \verb=#1= would default to ``world'' with the first parameter being \verb=#2=, otherwise, if there were two parameters, the first parameter would be \verb=#1= like normal. There is one special rule to follow, if the optional parameter is being used, then it must me enclosed by brackets rather than braces. If using the macro defined above, to have the same output as the one used in the previous section, the following macro would be used. \verb=\hello[Sarah]{Bryce}=. There ways to define macros with more than one optional parameter which is covered in \hyperref[section:programming/macros/otherWays]{Other Ways to Define Macros}.
\subsubsection{Macro Expansion Order}
In general, \LaTeX{} will expand macros in the order that it sees them. This means that if you have a macro as an argument to another macro, that second macro will see a macro as its argument rather than whatever that macro expands to. This is useful is some cases. For example, I have shown macros in this document so far using the \verb=\verb= command. If \LaTeX{} were to expand in a different order, the commands I wish to show you would be expanded before the verb command changed them to plain typewriter text. \par
There are some cases where this is not wanted. Perhaps you are trying to define a macro with a name based on the output of another macro. The explanation for how this works will be in the next section but it is important to note that this involves another macro which must be expanded before the macro to define the macro is expanded. It's easy to tell \LaTeX{} to expand the macro after the next one first using the \verb=\expandafter= command. It is also possible to chain these together, if \verb=\expandafter= is the target of the first \verb=\expandafter= command, then it will expend an even further macro first. This can be chained for quite a while before memory runs out. \par
Another important place where this expansion order comes into play is with macro definition itself. When a macro is defined, nothing in its definition is expanded. This means that if you define a macro as essentially an alias for another macro and then redefine the other macro, your original definition will change too. There are various ways to get around this which is covered later. \par
There is one issue that you may come across. Not all macros are expandable. The main reason for this is if it is a raw \TeX command which does not expand into anything but simply preform an action. Few of these have been discussed so far but they will come up later. There are a few other packages which have specific commands that have these issues as well. For example, the \texttt{xString} package states this in the documentation \texttt{The macros of this package are not purely expandable}. That package does provide ways to get around that issue however.
\subsubsection{Dynamic Macro Names}
\subsubsection{Other Types of Macros}
\subsubsection{Other Ways to Define Macros} \label{section:programming/macros/otherWays}
\subsubsection{Internal Macro Definitions}

\subsection{Common Programming Features in LaTeX}
\subsubsection{Variables}
\subsubsection{Arrays}
\subsubsection{If Statements}
\subsubsection{Switch Statements}


\subsection{Advanced Features} \label{section:programming/advancedFeatures}
\subsubsection{Category Codes}
\subsubsection{Active Characters}
\subsubsection{Counters}
\subsubsection{Iteration}
\subsubsection{Computation}
\subsubsection{Working With Strings}
\subsubsection{Hooks}
\subsubsection{Command Line Parameters}
\subsubsection{Plain \TeX} \label{section:programming/advancedFeatures/plainTeX}


\subsection{Synthesising \LaTeX{} From Other Languages}


\subsection{Calling External Programs}


\subsection{Programming Tikz Pictures}
\subsubsection{Nodes}
\subsubsection{Links}
\subsubsection{Styles}
\subsubsection{Recursion}


\subsection{Writing Packages}


\subsection{Writing Classes}


\subsection{Breaking \LaTeX}
\subsubsection{Issues With Online \LaTeX{} Compilers}
\subsubsection{Changing Everything in The Middle of The Document}
\subsubsection{Obfuscation}